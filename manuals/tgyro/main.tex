\documentclass[12pt]{article}
\usepackage{amsmath}
\usepackage{graphicx}
\usepackage{algorithmicx}
\usepackage{algpseudocode}

%--------------------------------------
\newcommand\df{\delta\hskip -1pt f}
\newcommand\chigb{\chi_{\rm GB}}
\newcommand\qgb{Q_{\rm GB}}
\newcommand\vpr{V^\prime}
\newcommand\qhat{\widehat{Q}}
\newcommand\jhat{\widehat{{\cal J}}}
\newcommand{\sj}{{\sigma,j}}
\newcommand{\sspjjp}{{\sigma\sigma^\prime,jj^\prime}}
\newcommand{\spjp}{{\sigma^\prime,j^\prime}}
\newcommand{\spj}{{\sigma^\prime,j}}
%--------------------------------------

\begin{document}
\begin{center}
{\bf TGYRO iteration scheme notes}\\
{\sl J. Candy}
\end{center}

\subsection{Transport equations}

We restrict attention in this note to the steady-state transport 
problem.  In this case, the transport equations for ion and 
electron energy flux are written
%
\begin{align}
\frac{1}{\vpr(r)} \frac{\partial}{\partial r} \left[ 
\vpr(r) Q_i(r) \right] = &~(1-\lambda) S_\alpha + S_{\rm exch}^{ei} 
  + S_{\rm aux}^i \; , \\ 
\frac{1}{\vpr(r)} \frac{\partial}{\partial r} \left[ 
\vpr(r) Q_e(r) \right] = &~\lambda S_\alpha - S_{\rm exch}^{ei} 
 + S_{\rm aux}^e - S_{\rm rad} \; . 
\end{align}
%
Here, the energy fluxes are the taken to be the sum of neoclassical 
transport (Hinton-Hazeltine analytic theory or NEO simulation) and 
turbulent transport (IFS-PPPL, QFM, TGLF or GYRO simulation):
%
\begin{align}
Q_i = & ~Q_i^{\rm Neo} + Q_i^{\rm Turb} \\ 
Q_e = & ~Q_e^{\rm Neo} + Q_e^{\rm Turb} 
\end{align}
%
At present we account for sources, radiation and exchange:
%
\begin{align*}
S_\alpha \rightarrow &~\text{Thermonuclear source} \\
S_{\rm exch}^{ei} \rightarrow &~\text{Electron-ion energy exchange} \\
S_{\rm aux}^i \rightarrow &~\text{Ion Auxilliary heating} \\
S_{\rm aux}^e \rightarrow &~\text{Electron Auxilliary heating} \\
S_{\rm rad} \rightarrow &~\text{Electron radiation} \\
\lambda \rightarrow &~\text{Fraction of heating to electrons}
\end{align*}
% 
\subsection{Some comments regarding units}

In TGYRO, we have found it more convenient to use CGS units
rather than employing some variant of the more popular 
dimensionless normalizations.  Thus, we have
%
\begin{align}
{\rm Source}: \quad &~ S \sim \frac{{\rm erg}}{{\rm cm}^3 \, {\rm s}} \\
{\rm Energy Flux}: \quad &~ Q \sim \frac{{\rm erg}}{{\rm cm}^2 \, {\rm s}} \\
{\rm Power}: \quad & P \sim  \frac{{\rm erg}}{{\rm s}} \rightarrow 
  \int_0^r dx \, V^\prime(x) S(x) 
\end{align}

\subsection{Solution strategy}

Rather than solving the equations directly, we prefer to solve 
the volume-integrated form of the equation so that we can 
deal directly with the fluxes:
%
\begin{align}
Q_i^T(r) \doteq \frac{1}{\vpr(r)} & ~\int_0^r dx \, \vpr(x) \left[
  (1-\lambda) S_\alpha + S_{\rm exch}^{ei} + S_{\rm aux}^i \right] \\ 
Q_e^T(r) \doteq \frac{1}{\vpr(r)} & ~\int_0^r dx \, \vpr(x) \left[ 
  \lambda S_\alpha - S_{\rm exch}^{ei} + S_{\rm aux}^e - S_{\rm rad} \right]  
\end{align}
%
The result is a curious system which depends on both the temperatures 
and the temperature gradients:
%
\begin{align}
Q_i(z_i,z_e,T_i,T_e) - Q_i^T(T_i,T_e) = &~0 \\
Q_e(z_i,z_e,T_i,T_e) - Q_e^T(T_i,T_e) = &~0
\end{align}
%
where
%
\begin{equation}
z_i \doteq - \frac{a}{T_i} \frac{\partial T_i}{\partial r} 
\quad\mbox{and}\quad 
z_e \doteq - \frac{a}{T_e} \frac{\partial T_e}{\partial r}
\end{equation}
%
It is important to note the connection between profiles and 
gradients.  Specifically, if we enforce the following pedestal 
boundary conditions at $r=r_*$:
%
\begin{align}
T_\sigma(r_*) = &~T_\sigma^* \; .
\end{align}
%
Then the gradients $z_\sigma$ uniquely determine the 
temperature profiles, $T_\sigma$:
%
\begin{equation}
T_\sigma(r) = T_\sigma^* 
 \exp\left( \int_r^{r_*} dx \, z_\sigma(x) \right) \; .
\end{equation}

\subsection{Formulation on a discrete grid}

On a discrete grid $\{r_j\}$, the temperature profile can 
be approximately determined using the trapezoidal rule 
%
\begin{equation}
T_\sigma(r_{j-1}) = T_\sigma(r_j) \exp \left\{
 \left[ \frac{z_\sigma(r_j)+z_\sigma(r_{j-1})}{2} \right]
 \left[ r_j-r_{j-1} \right] \right\} \; .
\end{equation}
%
To put the problem into discrete form, we define a vector of 
independent variables (gradients) and functions (fluxes):
%
\begin{align}
z_\sj = &~ z_\sigma(r_j) \; , \\
Q_\sj = &~ Q_\sigma(r_j) \; , \\
Q^T_\sj = &~ Q^T_\sigma(r_j) \; .
\end{align}
%
Then, the equations to be solved are
%
\begin{equation}
\qhat_\sj = \qhat^T_\sj  \; .
\end{equation}
%
where a hat denotes gyroBohm normalization:
%
\begin{equation}
\qhat \doteq \frac{Q}{\qgb} 
\quad \text{where} \quad \qgb = n_e T_e c_s (\rho_s/a)^2 \; .
\end{equation}
%
The goal is to apply Newton's method in a way which is as accurate 
as possible while still minimizing evaluation of the expensive functions 
$Q_\sj$.  Operationally, we make the key assumption that the transport 
fluxes depend only locally on the gradients (which is approximately 
true when quantities are normalzied to the gyroBohm unit of flux), 
so that the Jacobian associated with $Q_\sj$ is block diagonal:
%
\begin{equation}
\qhat_\sj(z^0) - \qhat^T_\sj(z^0) 
 + \frac{\partial \qhat_\sj}{\partial z_\spj} \,\delta z_\spj 
 - \frac{\partial \qhat^T_\sj}{\partial z_\spjp} \, \delta z_\spjp
 = 0 \; .
\end{equation}
%
Above, we have used the shorthand $z \doteq \{z_\sj\}$ and 
$z^0 \doteq \{z^0_\sj\}$.  This can be written in terms of 
Jacobian matrices as
%
\begin{equation}
\jhat_{\sspjjp} \, \delta z_{\sigma^\prime,j^\prime} = 
 -\left[ \qhat_\sj(z^0) - \qhat^T_\sj(z^0) \right] \eta_\sj \; ,
\label{eq.newton}
\end{equation}
%
where 
%
\begin{equation}
\jhat_\sspjjp \doteq {\cal J}_{\sigma\sigma^\prime,jj} \delta_{jj^\prime}
 -{\cal J}^T_{\sspjjp}\; ,
\end{equation}
%
and the quantity $z^1 = z^0 + \delta z$ is the Newton 
update for the vector $z$  In Eq.~(\ref{eq.newton}), we have
introduced a {\it relaxation parameter} $\eta_\sj$. Note that 
this method generalizes 
to an arbitrary number of gradients and fluxes per gridpoint.  
In the case of three radial gridpoints, $\{r_1,r_2,r_3\}$, 
the Jacobian matrices have the explicit forms

\begin{equation}
{\cal J}_{\sspjjp} = \begin{pmatrix}
\displaystyle \frac{\partial \qhat_{i,1}}{\partial z_{i,1}} &
\displaystyle \frac{\partial \qhat_{i,1}}{\partial z_{e,1}} &
0 & 0 & 0 & 0 \medskip \\
\displaystyle \frac{\partial \qhat_{e,1}}{\partial z_{i,1}} &
\displaystyle \frac{\partial \qhat_{e,1}}{\partial z_{e,1}} &
0 & 0 & 0 & 0 \medskip \\
0 & 0 &
\displaystyle \frac{\partial \qhat_{i,2}}{\partial z_{i,2}} &
\displaystyle \frac{\partial \qhat_{i,2}}{\partial z_{e,2}} &
0 & 0  \medskip \\
0 & 0 &
\displaystyle \frac{\partial \qhat_{e,2}}{\partial z_{i,2}} &
\displaystyle \frac{\partial \qhat_{e,2}}{\partial z_{e,2}} &
0 & 0  \medskip \\
0 & 0 & 0 & 0 &
\displaystyle \frac{\partial \qhat_{i,3}}{\partial z_{i,3}} &
\displaystyle \frac{\partial \qhat_{i,3}}{\partial z_{e,3}} &
\medskip \\
0 & 0 & 0 & 0 &
\displaystyle \frac{\partial \qhat_{e,3}}{\partial z_{i,3}} &
\displaystyle \frac{\partial \qhat_{e,3}}{\partial z_{e,3}} &
\end{pmatrix}
\end{equation}
\begin{equation}
{\cal J}^T_{\sspjjp} = \begin{pmatrix}
\displaystyle \frac{\partial \qhat^T_{i,1}}{\partial z_{i,1}} &
\displaystyle \frac{\partial \qhat^T_{i,1}}{\partial z_{e,1}} &
\displaystyle \frac{\partial \qhat^T_{i,1}}{\partial z_{i,2}} &
\displaystyle \frac{\partial \qhat^T_{i,1}}{\partial z_{e,2}} &
\displaystyle \frac{\partial \qhat^T_{i,1}}{\partial z_{i,3}} &
\displaystyle \frac{\partial \qhat^T_{i,1}}{\partial z_{e,3}} 
\medskip \\
\displaystyle \frac{\partial \qhat^T_{e,1}}{\partial z_{i,1}} &
\displaystyle \frac{\partial \qhat^T_{e,1}}{\partial z_{e,1}} &
\displaystyle \frac{\partial \qhat^T_{e,1}}{\partial z_{i,2}} &
\displaystyle \frac{\partial \qhat^T_{e,1}}{\partial z_{e,2}} &
\displaystyle \frac{\partial \qhat^T_{e,1}}{\partial z_{i,3}} &
\displaystyle \frac{\partial \qhat^T_{e,1}}{\partial z_{e,3}} 
\medskip \\
\displaystyle \frac{\partial \qhat^T_{i,2}}{\partial z_{i,1}} &
\displaystyle \frac{\partial \qhat^T_{i,2}}{\partial z_{e,1}} &
\displaystyle \frac{\partial \qhat^T_{i,2}}{\partial z_{i,2}} &
\displaystyle \frac{\partial \qhat^T_{i,2}}{\partial z_{e,2}} &
\displaystyle \frac{\partial \qhat^T_{i,2}}{\partial z_{i,3}} &
\displaystyle \frac{\partial \qhat^T_{i,2}}{\partial z_{e,3}} 
\medskip \\
\displaystyle \frac{\partial \qhat^T_{e,2}}{\partial z_{i,1}} &
\displaystyle \frac{\partial \qhat^T_{e,2}}{\partial z_{e,1}} &
\displaystyle \frac{\partial \qhat^T_{e,2}}{\partial z_{i,2}} &
\displaystyle \frac{\partial \qhat^T_{e,2}}{\partial z_{e,2}} &
\displaystyle \frac{\partial \qhat^T_{e,2}}{\partial z_{i,3}} &
\displaystyle \frac{\partial \qhat^T_{e,2}}{\partial z_{e,3}} 
\medskip \\
\displaystyle \frac{\partial \qhat^T_{i,3}}{\partial z_{i,1}} &
\displaystyle \frac{\partial \qhat^T_{i,3}}{\partial z_{e,1}} &
\displaystyle \frac{\partial \qhat^T_{i,3}}{\partial z_{i,2}} &
\displaystyle \frac{\partial \qhat^T_{i,3}}{\partial z_{e,2}} &
\displaystyle \frac{\partial \qhat^T_{i,3}}{\partial z_{i,3}} &
\displaystyle \frac{\partial \qhat^T_{i,3}}{\partial z_{e,3}} 
\medskip \\
\displaystyle \frac{\partial \qhat^T_{e,3}}{\partial z_{i,1}} &
\displaystyle \frac{\partial \qhat^T_{e,3}}{\partial z_{e,1}} &
\displaystyle \frac{\partial \qhat^T_{e,3}}{\partial z_{i,2}} &
\displaystyle \frac{\partial \qhat^T_{e,3}}{\partial z_{e,2}} &
\displaystyle \frac{\partial \qhat^T_{e,3}}{\partial z_{i,3}} &
\displaystyle \frac{\partial \qhat^T_{e,3}}{\partial z_{e,3}} 
\end{pmatrix}
\end{equation}
%
An important quantity to measure after a Newton iteration is
the residual 
\begin{equation}
R^1_\sj = \frac{\left[\qhat_\sj(z^1)-\qhat^T_\sj(z^1)\right]^2}{
 \left[\qhat_\sj(z^1)\right]^2+\left[\qhat^T_\sj(z^1)\right]^2}
\end{equation}
%
If, after a Newton step, any $R^1_\sj > R^0_\sj$ is not 
reduced, some strategy must be adopted to modify the 
gradient vector $z^1$ and/or the target.  This strategy 
is under development.  Note that there are two distinct 
iterations: 
%
\begin{enumerate}
\item
A Newton iteration, which is rapidly convergent given 
that one is close to a root and the $\qhat$ are smooth 
functions, 
\item
A fixed-point iteration following the Newton iteration,
because the weak profile variation of $\qhat$ was 
ignored
\end{enumerate}
If the temperature dependence of $\qhat$ was included, there 
would be no fixed-point iteration component.

\subsection{Numerical implementation}

\subsubsection{Computation of the Jacobian}

We approximate the derivatives in the Jacobian 
matrix using a forward difference approximation
%
\begin{equation}
\frac{\partial \qhat_\sj}{\partial z_\spjp} \simeq 
\frac{\qhat_\sj (z_\spjp + \Delta z) -\qhat_\sj (z_\spjp)}{\Delta z} 
\end{equation}
%
A desireable feature of this approximation is that the 
iteration scheme, Eq.~(\ref{eq.newton}) if it converges, 
will converge to the exact root of the original equations 
without any influence of the finite-difference truncation 
error.

\subsubsection{Iteration logic}

If iteration proceeds such that all residuals $R_\sj$ 
decrease at each iteration, then $\eta_\sj = 1$ and no 
additional logic is required.  However, in practice, 
not all residuals will decrease.  In this case, we 
implement a residual management strategy.  Here is the
complete logic:

\begin{algorithmic}[1]
\ForAll{$\sj$}
\Comment Capped Newton step
\State $\delta z_\sj \gets -\left(\jhat^{-1}\right)_\sspjjp 
\left[ \qhat_\spjp(z^0) - \qhat^T_\spjp(z^0) \right] \eta_\spjp$
\If{$|\delta z_\sj| > \Delta z_{\rm max}$}
\State $\delta z_\sj \gets \text{sgn}({\delta z_\sj}) \, \Delta z_{\rm max}$ 
\State $z_\sj^1 \gets z_\sj^0 + \delta z_\sj$
\EndIf
\EndFor
\end{algorithmic}

\begin{algorithmic}[1]
\ForAll{$\sj$}
\Comment Manage increasing residual
\If{$R_\sj^1 > R_\sj^0$}
\State $z_\sj^1 \gets z_\sj^0$
\State $\eta_\sj \gets \eta_\sj/C_\eta$ 
\If{$\eta_\sj < 1/C_\eta^3$} 
\State $\eta_\sj \gets 0.75 \, C_\eta$
\EndIf
\Else
\State $\eta_\sj \gets 1$
\EndIf
\EndFor
\end{algorithmic}
%
Then, the residuals $R^1$ are recomputed and the 
Newton stepping continues.  
 
\bibliographystyle{gareport}
\bibliography{global}

\end{document}
